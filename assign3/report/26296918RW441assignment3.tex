\documentclass[10pt, conference]{IEEEtran}

\usepackage{cite}
\usepackage{url}
\usepackage{amsmath,amssymb,amsfonts}
\usepackage{algpseudocode}
\usepackage{algorithm}
\usepackage{graphicx}
\usepackage{textcomp}
\usepackage{xcolor}
\def\BibTeX{{\rm B\kern-.05em{\sc i\kern-.025em b}\kern-.08em
    T\kern-.1667em\lower.7ex\hbox{E}\kern-.125emX}}
\begin{document}

\title{Active Learning in Neural Networks\\

}

\author{\IEEEauthorblockN{DT Nicolay 26296918}
\IEEEauthorblockA{\textit{Computer Science Division} \\
\textit{Stellenbosch University}\\
Stellenbosch, South Africa \\
26296918@sun.ac.za}
}

\maketitle

\begin{abstract}
TODO
\end{abstract}

\begin{IEEEkeywords}
TODO
\end{IEEEkeywords}

\section{Introduction}
TODO

\section{Background}
%intro

\subsection{Passive Learning}
% - Traditional supervised learning paradigm
% - Stochastic Gradient Descent algorithm
% - Random sampling of training data
% - Advantages and limitations
% - Mathematical formulation of SGD updates

\subsection{Active Learning Paradigm}
% - Core concepts and motivation
% - Query strategy framework
% - Pool-based vs stream-based vs membership query synthesis
% - Theoretical advantages over passive learning

\subsection{Output Sensitivity Analysis}
% - Mathematical foundation of sensitivity analysis
% - How gradients indicate information content
% - Instance selection based on network sensitivity
% - Algorithm description and theoretical justification
% - Reference to SASLA.pdf concepts

\subsection{Uncertainty Sampling}
% - Uncertainty measures in classification vs regression
% - Entropy-based uncertainty for classification
% - Variance-based uncertainty for function approximation
% - Algorithm description and theoretical foundation
% - Reference to ALUS.pdf concepts



\section{Implementation}

\subsection{Neural Network Architecture}
% - Single hidden layer specification
% - Hidden unit determination process (overestimate strategy)
% - Activation function selection and justification
% - Cost function selection
% - Regularisation

\subsection{Active Learning Algorithms}
% - Detailed implementation of output sensitivity analysis
% - Detailed implementation of uncertainty sampling
% - Query selection mechanisms
% - Batch size considerations for active learning

\subsection{Software and Hardware Specifications}
% - Programming language and libraries used
% - Computational resources
% - Code repository reference (as required)



\section{Emperical Process}
\subsection{Dataset Selection and Preprocessing}
% - Description of at least 3 classification problems
% - Description of at least 3 function approximation problems  
% - Varying complexity justification
% - Data preprocessing steps with motivations
% - Train/validation/test splits

\subsection{Performance Criteria}
% - Classification metrics: accuracy, precision, recall, F1-score
% - Function approximation metrics: MSE, MAE, R²
% - Learning curve analysis
% - Sample efficiency measures
% - Statistical significance testing approach

\subsection{Experimental Design}
% - Cross-validation strategy
% - Multiple runs for statistical reliability
% - Control parameter optimization process
% - Hyperparameter tuning methodology
% - Fair comparison ensuring protocols

\subsection{Statistical Analysis Framework}
% - Hypothesis testing approach
% - Significance levels
% - Multiple comparison corrections
% - Effect size measurements

\section{Results \& Discussion}
% (50 marks - largest section, most critical)
\subsection{Dataset Characteristics}
% - Detailed description of each dataset
% - Complexity analysis and classification difficulty
% - Baseline performance metrics

\subsection{Hyperparameter Optimization Results}
% - Optimal hidden unit numbers for each problem
% - Optimal penalty coefficients
% - Control parameter values with justifications
% - Convergence analysis

\subsection{Classification Problem Results}
% - Performance comparison across all three approaches
% - Learning curves showing sample efficiency
% - Statistical significance of differences
% - Problem-specific performance analysis
% - Tables and figures with proper captions

\subsection{Function Approximation Results}
% - Detailed results for each regression problem
% - Comparison of convergence rates
% - Sample efficiency analysis
% - Performance variance across runs

\subsection{Comparative Analysis}
% - Overall ranking of approaches
% - Problem-dependent performance patterns
% - Sample efficiency comparison
% - Computational cost analysis
% - Advantages and disadvantages of each approach

\subsection{Discussion of Findings}
% - Interpretation of results in context of theory
% - Explanation of why certain methods performed better
% - Limitations of the study
% - Practical implications for real-world applications

\section{Conclusions}
TODO

%\bibliographystyle{IEEEtran}
%\bibliography{references}

\end{document}
